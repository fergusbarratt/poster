\documentclass[a0,portrait]{a0poster}
%%%Load packages
\usepackage{multicol} 			%3-column layout
\usepackage[left=6cm,right=2cm,bottom=-2cm,top=0cm]{geometry}			%Reset margins
\usepackage[T1]{fontenc}			%Need for gtamac fonts
\usepackage{textcomp}
\usepackage{mathpazo}			%Load palatino font & pazo math
\usepackage{color}				%Needed for colour boxes & coloured text
\usepackage{amsmath}


%%%Define colours and lengths
\definecolor{headingcol}{rgb}{1,1,1}		%Colour of main title
\definecolor{fillcol}{rgb}{1.0,1.0,1}			%Fill-colour of box
\definecolor{boxcol}{rgb}{0.3,0.5,0.5}		%Edge-colour of box and top banner
\definecolor{author}{rgb}{0.9, 0.9, 0.9}
\fboxsep=1cm							%Padding between box and text
\fboxrule=3mm							%Width of box outline
\renewcommand{\rmdefault}{ppl}			%Reset serif to Palatino
\setlength{\columnsep}{3cm}				%Set spacing between columns
%Uncomment for lines as column separators:
%\setlength{\columnseprule}{1pt}


%%%Format title
\makeatletter							%Needed to include code in main file
\renewcommand\@maketitle{%
\null									%Sets position marker
{
    \vspace{3cm}
\includegraphics[width=10cm]{kings-logo}\vspace{-12cm}
\vspace{2cm}
\color{headingcol}\rmfamily\huge\bf		%Set title font and colour
\vskip 2cm
\hspace{18cm}\begin{minipage}{0.4\textwidth} \centering
\@title \par
\end{minipage}}%
\vskip 0.6em%
{
\color{author}\sffamily\Large\bf				%Set author font and colour
\lineskip .5em%
\hspace{13cm}\begin{minipage}{0.5\textwidth}
    \centering

\@author
\par
\vspace{1.1cm}
\end{minipage}}%
\vskip 1cm
\par
}
\makeatother

\newsavebox\envbox 					%Define name for boxes used

%%%Define "Section" environment for framed boxes
%%%Usage: \begin{Section}{Name} blah blah blah \end{Section}
\newenvironment{Section}[1]				%Environment takes one argument
%%%Opening
{
\par 
\flushleft
\colorbox{boxcol}{% 						%Draws solid colour box around title
\sffamily\large\bf\color{headingcol}> \color{white} #1%Typesets section name
\hspace{0.5cm}}
\par\nobreak 
\nointerlineskip 						%Fits title snugly above box (no gap)
\setlength\parskip{-1pt}					%Even snugger
\begin{lrbox}\envbox						%Opens box environment
\begin{minipage}{0.95\columnwidth}		%Opens minipage environment for section contents
}
%%%Closing
{\par
\end{minipage}\end{lrbox}				%Close minipage and box
\fcolorbox{boxcol}{fillcol}{\usebox\envbox}	%Draw box with contents frame colour: boxcol, fill colour: fillcol
\vspace{1cm}							%Add spacing below box
} 
\newenvironment{Section2}[1]				%Environment takes one argument
%%%Opening
{
\par 
\flushleft
\colorbox{boxcol}{% 						%Draws solid colour box around title
\sffamily\large\bf\color{headingcol}> \color{white} #1%Typesets section name
\hspace{0.5cm}}
\par\nobreak 
\nointerlineskip 						%Fits title snugly above box (no gap)
\setlength\parskip{-1pt}					%Even snugger
\begin{lrbox}\envbox						%Opens box environment
    \begin{minipage}{2.04\columnwidth}		%Opens minipage environment for section contents
}
%%%Closing
{\par
\end{minipage}\end{lrbox}				%Close minipage and box
\fcolorbox{boxcol}{fillcol}{\usebox\envbox}	%Draw box with contents frame colour: boxcol, fill colour: fillcol
\vspace{1cm}							%Add spacing below box
} 
\newenvironment{Section3}
%%%Opening
{
\par 
\flushleft
\par\nobreak 
\nointerlineskip 						%Fits title snugly above box (no gap)
\setlength\parskip{-1pt}					%Even snugger
\begin{lrbox}\envbox						%Opens box environment
    \begin{minipage}{2.04\columnwidth}		%Opens minipage environment for section contents
}
%%%Closing
{\par
\end{minipage}\end{lrbox}				%Close minipage and box
\fcolorbox{boxcol}{fillcol}{\usebox\envbox}	%Draw box with contents frame colour: boxcol, fill colour: fillcol
\vspace{1cm}							%Add spacing below box
} 
\newenvironment{Section4}
%%%Opening
{
\par 
\flushleft
\par\nobreak 
\nointerlineskip 						%Fits title snugly above box (no gap)
\setlength\parskip{-1pt}					%Even snugger
\begin{lrbox}\envbox						%Opens box environment
    \begin{minipage}{0.9843\columnwidth}		%Opens minipage environment for section contents
}
%%%Closing
{\par
\end{minipage}\end{lrbox}				%Close minipage and box
\fcolorbox{boxcol}{fillcol}{\usebox\envbox}	%Draw box with contents frame colour: boxcol, fill colour: fillcol
\vspace{1cm}							%Add spacing below box
} 
\newenvironment{Section5}
%%%Opening
{
\par 
\flushleft
\par\nobreak 
\nointerlineskip 						%Fits title snugly above box (no gap)
\setlength\parskip{-1pt}					%Even snugger
\begin{lrbox}\envbox						%Opens box environment
    \begin{minipage}{0.636\columnwidth}		%Opens minipage environment for section contents
}
%%%Closing
{\par
\end{minipage}\end{lrbox}
\fcolorbox{boxcol}{fillcol}{\usebox\envbox}\hspace{-1.6cm}	%Draw box with contents frame colour: boxcol, fill colour: fillcol
\vspace{1cm}							%Add spacing below box
} 
\usepackage{physics}
\usepackage{graphicx}
\graphicspath{ {images/} }
\usepackage{caption}
\usepackage{array}
 

\title{Understanding Dissipative Control of Quantum Dynamics from Biased Trajectory Ensembles}

\author{Fergus Barratt (fergus.barratt@kcl.ac.uk) \\Supervisors: Andrew Green \& Peter Sollich}

\begin{document}
\pagenumbering{gobble}

\hspace{-4.1cm}								%Align with edge of page, not margin
\colorbox{boxcol}{							%Coloured banner across top
\begin{minipage}{1189mm}					%Minipage for title contents
%\vspace{-13cm}							%Shift up over header image
\maketitle
\end{minipage}}
\vspace{1cm}

\begin{multicols}{3}							%Use 3-column layout
\raggedcolumns							%Don't stretch contents vertically
\Large

%%%Column1
\begin{Section}{Introduction}
\begin{itemize}
    \item Technological potential of quantum systems due to exponential state space scaling.
    \item Environmental decoupling challenges effective use of quantum technology.
    \item Tools from classical statistical mechanics?
\end{itemize}
\end{Section}

\begin{Section}{Trajectory Ensembles}
            \includegraphics[width=1\linewidth]{dyn_phase_transition}
\captionof{figure}{Dynamical phase transition in KCM. $\mathcal{K}(s)$ is the first derivative of $\psi_K(s)$ the dynamical free energy\cite{Garrahan2007}}
    \begin{itemize}
        \item Ensembles of trajectories - like thermal ensembles. Partition function:~\cite{Garrahan2007}.
            \begin{equation}
                \mathcal{Z}_A(s, t) = \sum_A \Omega_{dyn}(A, t)e^{-sA}
            \end{equation}

        \item Markovian, paths biased (like canonical ensemble) by time extensive observable A (E) with strength s ($\beta$). $\mathbb{W} \rightarrow \mathbb{W}_A$ in
\begin{equation}
    \pdv{\vec{P}_A}{t} = \mathbb{W}_A(s) \vec{P}_A,
\end{equation}
leads to
\begin{equation}
    \mathcal{Z}_A(s, t) \sim e^{t \psi_A(s)}
\end{equation}
with $\psi_A(s)$ largest eigenvalue of $\mathbb{W}_A(s)$: \emph{dynamical free energy}.
\item Singularities in dynamical free energy $\rightarrow$ \emph{dynamical phase transitions}: different dynamical behaviour i.e. transition to chaos, jamming in glasses.
\item Dynamical phase transitions in \emph{kinetically constrained models} (KCMS) of glass formers (fig. \ref{fig:KCM})

\end{itemize}
\includegraphics[width=\linewidth]{kcm}
\captionof{figure}{Schematic of site of Kinetically Constrained Model. Site $i$ (blue) transitions from $n_i \rightarrow 1-n_i$ with rate $W(n_i \rightarrow 1-n_i) = C(\{n_j\}) \frac{e^{\beta(n_i-1)}}{1+e^{-\beta}}$, where $C(\{n_j\})$ is a function only of the values of sites $n_j$ (red)} \label{fig:KCM}
\end{Section}

\columnbreak

%%%Column 2

\begin{Section2}{Aims}
    \begin{itemize}
        \item Classical glasses $\rightarrow$ KCMs $\rightarrow$ \emph{biased trajectory ensembles}.
        \item Keldysh theory $\rightarrow$ dissipation: analogous to bias in trajectory ensembles?
        \item Consider adiabatic quantum computation. Spin chains $\rightarrow$ Matrix Product States
        \item Transition from usable quantum resources (i.e. entanglement) to absence.
    \end{itemize}

\end{Section2}

\begin{Section}{Matrix Product States}

\begin{itemize}
    \item Interesting bit of most Hilbert spaces small, permits efficient numerics.
    \item Many-body quantum state 
\begin{equation}
    \ket{\psi} = \sum_{\sigma_1, \ldots, \sigma_L} C_{\sigma_1, \ldots, \sigma_L} \ket{\sigma_1, \ldots, \sigma_L}, \nonumber
\end{equation}
\item MPS decomposition ($\Lambda$ diagonal)
    \begin{equation} \ket{\psi} = \sum_{\sigma_1, \ldots, \sigma_L} \Gamma^{\sigma_1}\Lambda_1 \ldots \Lambda_{L-1}\Gamma^{\sigma_L} \ket{\sigma_1, \ldots, \sigma_L }\nonumber
    \end{equation}
\item Important bit of Hilbert space spanned by MPS with $\Lambda$ truncated to D dims.
\item D grows with time evolution. Project back to D subspace - classical EOM for variational parameters. 
    \\
    \\
\includegraphics[width=\linewidth]{D}
\captionof{figure}{Size of subset of Hilbert space $H$ with bond dimension $D$ grows with $D$}
\vspace{1cm}

\end{itemize}
\end{Section}

\columnbreak

%%%Column 3

\flushleft\fcolorbox{white}{fillcol}{\usebox\envbox}
\vspace{11.5cm}
\begin{Section}{Non-Equilibrium QFT: Keldysh Theory}
\begin{itemize}
    \item Evolve the state forward in time $(-\infty \rightarrow \infty)$, rewind $\infty \rightarrow -\infty$: fig. \ref{fig:keldysh}
    \item Separate fields on forward and backwards contours ($\phi^+, \phi^-$)
    \item Construct the path integral $\rightarrow$ redundancy in the Green's Functions.
    \item Combinations such that $\ev{\phi^{cl}\phi^{cl}}$ = 0
    \begin{equation}
        \phi^q = \phi^+ - \phi^- \qquad \phi^{cl} = \phi^+-\phi^-
    \end{equation}
    \item Introduce a bath interacting with system, integrate out bath, generate semi-classical path integrals for quantum systems.  
\item Distribution over paths. Connected to the classical trajectory ensemble?
        \includegraphics[width=\linewidth]{spins1}
    \captionof{figure}{Trajectories of effective classical spins of qualitatively different dynamical forms~\cite{Crowley2016}}
    \vspace{0.3cm}
    \vspace{0.25cm}
\end{itemize}

\end{Section}
%\bibliographystyle{plain}
%\bibliography{halobib}

\end{multicols}
%%%Closing

\vspace{-38.5cm}
    \flushright
\begin{Section5}
\begin{minipage}{1\linewidth}
    \vspace{0.2cm}
    \centering
    \includegraphics[width=1.16\linewidth]{mps}
 \captionof{figure}{5-site Open Boundary Condition MPS in $\Gamma\Lambda$ notation}
\end{minipage}
\begin{minipage}{1\linewidth}
        \centering
\includegraphics[width=1\linewidth]{keldysh}\\
\captionof{figure}{Keldysh Contour}\label{fig:keldysh}
\end{minipage}
\end{Section5}
\vfill
\begin{Section5}
    \bibliography{ref}{}
    \bibliographystyle{abbrv}
\end{Section5}\hspace{-0.2cm}
\vspace{1cm}
\end{document}
